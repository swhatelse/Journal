\documentclass[smallextended]{svjour3} 
 \usepackage[T1]{fontenc}
\usepackage[utf8]{inputenc}
%\usepackage{fixltx2e}
\usepackage{ifthen,figlatex}
\usepackage{longtable}
\usepackage{float}
\usepackage{wrapfig}
\usepackage{subfigure}
\usepackage{graphicx}
\usepackage[export]{adjustbox}
\usepackage{xspace}
\usepackage{amsmath,amssymb}
\usepackage[french, frenchb]{babel}
\AtBeginDocument{
\definecolor{pdfurlcolor}{rgb}{0,0,0.6}
\definecolor{pdfcitecolor}{rgb}{0,0.6,0}
\definecolor{pdflinkcolor}{rgb}{0.6,0,0}
\definecolor{light}{gray}{.85}
\definecolor{vlight}{gray}{.95}
}
%\usepackage[paper=letterpaper,margin=1.61in]{geometry}
\usepackage{url} \urlstyle{sf}
\usepackage[normalem]{ulem}
\usepackage{todonotes}
\usepackage[colorlinks=true,citecolor=pdfcitecolor,urlcolor=pdfurlcolor,linkcolor=pdflinkcolor,pdfborder={0 0 0}]{hyperref}
\usepackage[round-precision=3,round-mode=figures,scientific-notation=true]{siunitx}
% \usepackage{minted}
% \usepackage{verbments}
% \usepackage{verbatim}
% \usepackage{alltt}

  \usepackage{graphicx}
  \usepackage{hyperref}
\date{\today}
\title{}
\hypersetup{
  pdfkeywords={},
  pdfsubject={},
  pdfcreator={}}
\begin{document}

\newcommand{\AL}[2][inline]{\todo[color=green!50,#1]{\sf \textbf{AL:} #2}\xspace}
\newcommand{\LS}[2][inline]{\todo[color=green!50,#1]{\sf \textbf{LS:} #2}\xspace}

\let\oldcite=\cite
\renewcommand\cite[2][]{~\ifthenelse{\equal{#1}{}}{\oldcite{#2}}{\oldcite[#1]{#2}}\xspace}
\let\oldref=\ref
\def\ref#1{~\oldref{#1}\xspace}
\def\ie{i.e.,\xspace}
\def\eg{e.g.,\xspace}
\def\qrmspu{\texttt{QRM\_StarPU}\xspace}
\sloppy

\title{Insert your title here%\thanks{Grants or other notes
%about the article that should go on the front page should be
%placed here. General acknowledgments should be placed at the end of the article.}
}
%\subtitle{Do you have a subtitle?\\ If so, write it here}

%\titlerunning{Short form of title}        % if too long for running head

\author{Steven QUINITO MASNADA  \\ \and \\
        Supervised by : Arnaud LEGRAND \and Luka STANISIC  %if many names separate them with \and.
}

%\authorrunning{Short form of author list} % if too long for running head

\institute{F. Author \at
              first address \\
              Tel.: +123-45-678910\\
              Fax: +123-45-678910\\
              \email{fauthor@example.com}           %  \\
%             \emph{Present address:} of F. Author  %  if needed
           \and
           S. Author \at
              second address
}

\date{June 2015}
% The correct dates will be entered by the editor

\maketitle


\begin{abstract}
Dans le domaine des supercalculateurs, la course à la performance est
un point crucial. Actuellement, le calculateur le plus puissant (le
TianHe-2) est capable d'effectuer environ 33.86 Peta d'opérations
flotantes par secondes. Cependant cette course est freinée par un
facteur qui prend désormais d'une importance capitale, le coût
énergétique. En effet, reprennons l'exemple du supercalculateur
chinois, la consommation du TianHe-2 atteint presque les 18MW et
avec la génération exascale la consommation estimée sera entre 20MW
et 40MW. Dans l'état des fait, ce n'est pas réalisable et pour
pouvoir atteindre l'exaflops, il nécessaire d'optimiser d'autres
points que la puissance des puces. Evidemment des optimisations
peuvent être faites au niveau matériel afin de réaliser des
composants à hautes efficacités énergétiques. On peut également
optimiser le rendement en utilisant au mieux les capacités du
matériel. Cette optimisation ce fait donc du côté logiciel et pour
cela il nous faut  envisager un changement de méthode programmation,
c'est cette dernière que nous allons étudier. L'objectif de mon
stage au sein de l'équipe MESCAL, sous la tutelle d'Arnaud Legrand,
est donc de tenter de mesurer le gain d'une telle solution. 

Pour cela nous allons, dans une première partie, voir comment est
effectuée en générale la programmation en HPC, quels sont différents
les standards et pourquoi nous nous sommes concentrés sur MPI. Nous
discuterons ensuite du principe et de l'intérêt d'un nouveau
paradigme de programmation et de la librairie StarPU. Nous
constaterons ensuite que malgrès les apports de cette méthodes des
difficultés subsites et les mesures peuvent-être compliquées a
effectuées. C'est pourquoi dans une seconde partie, nous étudierons
les différents approches pour évaluer les performances
d'applications HPC et nous justifierons notre choix pour la
simulation/émulation et en particulier pour l'outils Simgrid. Dans
une troisième partie nous examinerons en détail Simgrid et StarPU
ainsi que les différents problèmes que nous avons rencontrés. Dans
une quatrième partie, nous verrons les méthodes employées. En
cinquième partie, nous verrons les modifications apportés à Simgrid
afin de pouvoir effectuer les mesures. Ensuite dans une sixième
partie, nous verrons comment ces changements ont été validés. Et
pout finir nous conclurons sur les résultats que nous avons réussit
à obtenir.
\end{abstract}

\section{Introduction}
\label{sec-1}

La majorité des supercalculateurs actuels sont des clusters
massivement parallèles et souvent de type
hétérogènes(CPU-GPU). De ce fait certains standard ce sont
imposés. 

Il y a tout d'abord la norme MPI (Message Passing Interface),
qui est une API de communication basée sur l'envoi et la
récéption de message. Elle réputée pour être performante et
portable, et elle est de plus haut niveau que les sockets.

Ensuite, il y a l'API OpenMP qui est une interface de
multihreading de plus haut niveau de PThread\ldots{}

Enfin, l'API CUDA permet tirer partie de la puissance de calculer
des GPU. Mais pour cela il est nécessaire de spécifier
explicitement de ce que l'on veut envoyer aux GPUs et on doit
également gérer la synchronisation 

L'objectif étant optimiser le rendemment d'une application afin
que cette dernière tire partie de toute la puissance disponible, il faut
faire en sorte d'occuper au maximum le plus d'unités de calcul possible.
Le problème est que l'on se retrouve à devoir utiliser plusieurs
paradigme à la fois ce qui complique grandement la
programmation.

Généralement, on procède soit en délèguant tout 
les calculs aux GPUs, laissant les CPUs en idle. Soit on réparti
la charge entre les CPUs et les GPUs de manière complètement
statique. L'inconvéniant est que la mise en pratique très
difficile car il est hardu de trouver un bon équilibrage. 

Cependant même si l'on arrive à équilibrer les charges
correctemment, on peut avoir des cas où certaines unités de
caluls ne sont pas occupées alors qu'elles le pourraient. Cela se
produit quand par exemple lorsque certaines unités de calculs
attendent la terminaisons de certain traitements alors que
d'autres auraient put être effectuer en attendant. Cela est dû au
fait que l'exécution soit statique et induit un idle time
artificiel. De plus cette solution n'est pas portable car le
découpage des traitements ce fait en fonction de la plateforme
cible.

La solution serait donc d'avoir une gestion dynamique des
charges. Mais cela s'avère bien plus ardue, voir impossible
à réaliser directement avec ces méthodes de programmation. Alors
essayons en une autre.

La librairie StarPU est un système runtime qui permet une
répartition des traitements de manière dynamique et opportuniste. 
Pour ce faire elle introduit un nouveau paradigme basé sur les
tâches. StarPU génère un graphe de dépendance permettant
d'optimiser le l'ordonnancement de ces dernières.

La première version de StarPU a été conçu spécialement pour des
architectures hybrides. Une version récente (StarPU MPI) a été
réalisée pour bénéficier d'un ordonnance et d'une exécution qui
soit à la fois dynamique et opportunistes dans un contexte
distribuée afin de répartir la charge entre les différents
noeuds.

Nous allons donc voir comment évaluer les performances
d'applications basés sur StarPU MPI.

\section{État de l'art}
\label{sec-2}
\subsection{sparse solvers}
\label{sec-2-1}
MUMPS, QR-MUMPS, \ldots{}.
\subsection{Runtimes}
\label{sec-2-2}
\begin{itemize}
\item Dague, Quark, StarSS
\item StarPU has the right level of abstraction and organization
(synchronization, software architecture), is stable and has good
performances
\end{itemize}
\subsection{Simulation for HPC}
\label{sec-2-3}
Most tools focud on simulation of MPI applications and
are based on trace replay. SimGrid provides thread-like API, which
simplifies application porting.
\subsection{Conclusion, we use the best of these three domains}
\label{sec-2-4}
\section{Presentation of SimGrid/StarPU/Qrm\_Starpu}
\label{sec-3}
\subsection{Qrm\_Starpu}
\label{sec-3-1}
How does it work. DAG, eliminition tree, main kernels (INIT, CLEAN,
DO\_SUBTREE, GEQRT). A priori complexity of some of the kernels known by
the developpers.
\subsection{StarPU/SimGrid}
\label{sec-3-2}
Emulation/simulation/\ldots{} based on Europar and CCPE.
\section{Coarse grain Porting qrm\_starpu on top of StarPU/SimGrid:}
\label{sec-4}

Intro..

In order to run it with SimGrid, we had to do only few changes to
the initial qrm\_starpu code. Compilation process along with the
necessary environment variables had to be adapted to the simulation
mode of StarPU. Main \texttt{qrm\_test} program used to execute factorization
with qrm\_starpu had to be changed for the \texttt{starpu\_main} subroutine,
as SimGrid has its own main function and this is the easiest way to
avoid problems with having two mains.

This solution includes several major differences comparing to the
already published study of StarPU and SimGrid on dense linear
algebra applications\cite{Stanisic_CCPE}. Most notably, in
qrm\_starpu code there is a bigger number of different StarPU tasks
(also named \emph{kernels} in the rest of this paper) called with wide
variety of input parameters. When solving dense matrices, during an
application run, one kernel type (e.g., \emph{dgemm}) is always working
with the same block size, which makes its duration on the same
processing unit very stable. In qrm\_starpu factorization, the
amount of work that has to be done by a kernel depends hugely on its
input parameters and thus the duration will greatly vary. Hence, we
had to rework the task submission model to make sure all kernel
parameters are explicitly given to the StarPU so that they can be
propagated to SimGrid. Additionally, we were required to extend
tracing so that such kernel parameters are traced as well. This was
indispensable for analyzing and comparing traces for both real and
simulation executions.

On the other hand, so far we had worked with the qrm\_starpu
implementation that does not involve code running on GPUs, which
facilitates the simulation. The reason is that as there are no
critical need to model PCI contention and heterogeneity, thus there
are less elements and approximations to take into account with
SimGrid. Still, following very good results on hybrid machines with
dense linear algebra applications, we believe that it would not be
so difficult to extend this qrm\_starpu study to the fully
heterogeneous architectures.

Explain current memory allocation shortcoming..

\section{Conclusion}
\label{sec-5}
\begin{itemize}
\item Previous work for dense with GPUs with different schedulers but the
current version of the code is not ready yet and schedulers barely
matter for CPU only executions.
\item Still goes faster than real executions and does not require actual
machines.
\item Still allows for extrapolation, which can be useful
\item Open the path to performance studies that were not possible earlier
\end{itemize}

\section*{Acknowledgments}
This work is partially supported by the SONGS ANR project
(11-ANR-INFRA-13). Experiments presented in this paper were carried
out using the PLAFRIM experimental testbed, being developed under the
Inria PlaFRIM development action with support from LABRI and IMB and
other entities: Conseil Régional d'Aquitaine, Université de Bordeaux
and CNRS.

%\nocite{*}
\def\raggedright{}
\bibliographystyle{IEEEtran}
\bibliography{biblio}
\end{document}
%%% Local Variables:
%%% mode: latex
%%% TeX-master: t
%%% End:
